\documentclass{article}

\input{preamble.tex}

\begin{document}
\section{Introduction}
In our pursuit of mathematical truth, we encounter profound questions akin to the musings of poets on existence and the divine. Like the Persian poet pondering God's paradoxical existence, mathematicians grapple with the limitations of formal systems.



\begin{center}
\setlength{\fboxsep}{10pt}
\setlength{\fboxrule}{1pt}
\fbox{
\begin{minipage}{0.8\textwidth}
\centering
Oh sage, truth's guide, in wisdom's sway, \\
Amidst this puzzle, shed light, we say. \\

\vspace{5pt}

In quandaries deep, your insight share, \\
Resolve this riddle, make clear the air. \\

\vspace{5pt}

In void's embrace, where nothing's trod, \\
They say resides the form of God. \\

\vspace{5pt}

But if all else is naught, a void so vast, \\
Where can God reside, in what vast blast? \\
\end{minipage}
}\par\medskip
\textbf{\footnotesize{Poem by Mirdamad.}}
\end{center}

This timeless inquiry prompts contemplation on the paradoxes of mathematical reasoning. Gödel's incompleteness theorems shattered the illusion of absolute certainty, revealing that within any rich formal system, true statements exist that are unprovable within it.

This essay delves into Gödel's insights, tracing their origins and exploring their implications for mathematics and philosophy. Through this exploration, we aim to illuminate the intricate relationship between logic, truth, and the mysteries of mathematical existence.

\end{document}
